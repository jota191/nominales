\documentclass{article}

\usepackage{ amsmath }
\usepackage{ amssymb }
\usepackage{ bussproofs }
\usepackage{ multicol }
\usepackage{ enumitem }
\usepackage{ mathrsfs }
\usepackage[a4paper,margin=1in]{ geometry }

\newcommand\algRel{$\Rightarrow$}

\author{Juan García Garland}
\title{Resolution of Exercises - Nominal Techniques - 2019}
\date{}
\newcommand\fresh[2]{
  \relax
  \ifmmode
      {#1 \# #2}
  \else
      {$#1 \# #2$}
  \fi}
\newcommand\eqa[2]{
  \relax
  \ifmmode
      {#1 \approx_{\alpha} #2}
  \else
      $#1 \approx_{\alpha} #2$
  \fi}

%time to do it the right way

\newcommand\mathize[1]{
  \relax
  \ifmmode
      {#1}
  \else
      {$#1$}
  \fi}

\newcommand\Ax{\AxiomC}
\newcommand\Bin{\BinaryInfC}
\newcommand\Una{\UnaryInfC}

\newcommand\RL{\RightLabel}

\newcommand\abs[2]{\mathize{[#1] #2}}
\newcommand\lam[2]{\lambda\abs{#1}{#2}}

\newcommand\labstw{\text{\tiny $\approx$-abs2}}
\newcommand\lfsym{\text{\tiny $\approx$-func sym}}


\newcommand\LHS{\textrm{LHS}}
\newcommand\RHS{\textrm{RHS}}

\newcommand{\doubl}[1]
           {\left\lVert#1\right\rVert}
           \newcommand{\simpl}[1]
                      {\left|#1\right|}


\newcommand\assumption[1]{\AXC{}\doubleLine\UIC{#1}}
\begin{document}

\EnableBpAbbreviations

\maketitle

\section*{Exercise 1}

\begin{multicols}{2}
\subsection*{a)} Valid.
\begin{prooftree}
  \AXC{}
  \RightLabel{\tiny \#atom}
  \UIC{\fresh{x}{y}}
  \AXC{}
  \RightLabel{\tiny $\approx$-atom}
  \UIC{\eqa{x}{x}}
  \RightLabel{$\equiv$}
  \UIC{\eqa{x}{(x \: y) \cdot y}}
  \RightLabel{$\labstw$}
  \BIC{\eqa{\abs{x}{x}}{\abs{y}{y}}}
  \RightLabel{$\lfsym$}
  \UIC{\eqa{\lam{x}{x}}
          {\lam{y}{y}}}
\end{prooftree}

Note that in the application of the $\equiv$ ``rule'' there is no
actual rule applied. A permutation applied to a ground term is
syntactically equal to the result, there is no suspention here
(obviously, since there are no variables). Sometimes we will use this
extra step to make the applied permutation explicit, for clarity. We
named inference rules according to \cite{URBAN2004473}. We omit names
from now on since the system is syntax directed.

\subsection*{b)} Valid.

\begin{prooftree}
  \AXC{}
  \UIC{\fresh{x}{\abs{x}{y}}}
  \UIC{\fresh{x}{\lam{x}{y}}}
  \AXC{}
  \UIC{\eqa{{x}}{{x}}}
  \UIC{\eqa{\abs{y}{x}}{\abs{y}{x}}}
  \UIC{\eqa{{\lam{y}{x}}}{{\lam{y}{x}}}}\RightLabel{$\equiv$}
  \UIC{\eqa{{\lam{y}{x}}}{(x\: y) \cdot {\lam{x}{y}}}}
  %\labstw
  \BIC{\eqa{\abs{x}{\lam{y}{x}}}{\abs{y}{\lam{x}{y}}}}
  %\lfsym
  \UIC{\eqa{\lam{x}{\lam{y}{x}}}{\lam{y}{\lam{x}{y}}}}
\end{prooftree}


\subsection*{c)} Not valid.

\begin{prooftree}
  \AXC{?}
  \UIC{\fresh{x}{Y}}
  \AXC{?}
  \UIC{\eqa{X}{(x\: y) \cdot Y}}
  \BIC{\eqa{\abs{x}{X}}{\abs{y}{Y}}}
  \UIC{\eqa{\lam{x}{X}}{\lam{y}{Y}}}
\end{prooftree}

We are stuck. Open branches cannot be closed. Recall that the
deduction system is syntax directed. This implies that this situation
can only occur in a non valid judgement. Otherwise we could end in
open branches due to bad decisions when making the proof.

\vfill

\subsection*{d)}Not valid.
\begin{prooftree}
  \AXC{?}
  \UIC{\fresh{x}{Y}}
  \AXC{?}
  \UIC{\fresh{x}{X}}
  \AXC{?}
  \UIC{\fresh{y}{X}}
  \BIC{\eqa{X}{(x\: y) \cdot X}}
  \BIC{\eqa{\abs{x}{X}}{\abs{y}{X}}}
  \UIC{\eqa{\lam{x}{X}}{\lam{y}{X}}}
\end{prooftree}


\subsection*{e)} Not Valid.

\begin{prooftree}
  \AXC{}\doubleLine
  \UIC{\fresh{x}{X}}
  \doubleLine
  \AXC{}
  \UIC{\fresh{x}{X}}
  \AXC{?}
  \UIC{\fresh{y}{X}}
  \BIC{\eqa{X}{(x\: y) \cdot X}}
  \BIC{\eqa{\abs{x}{X}}{\abs{y}{X}}}
  \UIC{\eqa{\lam{x}{X}}{\lam{y}{X}}}
\end{prooftree}

We use a double line when a branch is closed by an assumption from the
set of hypothesis. In this case $\fresh{y}{X}$ cannot be proved and we
are stuck. But leaves with $\fresh{x}{X}$ were closed.


\subsection*{f)} Valid.
\begin{prooftree}
  \AXC{}\doubleLine
  \UIC{\fresh{x}{X}}
  \UIC{\fresh{x}{s(X)}}
  \AXC{}\doubleLine
  \UIC{\fresh{x}{X}}
  \AXC{}
  \doubleLine
  \UIC{\fresh{y}{X}}
  \BIC{\eqa{X}{(x\: y) \cdot X}}
  \UIC{\eqa{s(X)}{s((x\: y) \cdot X)}}
  \UIC{\eqa{s(X)}{(x\: y) \cdot s(X)}}
  \BIC{\eqa{\abs{x}{s(X)}}{\abs{y}{s(X)}}}
  \UIC{\eqa{\lam{x}{s(X)}}{\lam{y}{s(X)}}}
\end{prooftree}


\subsection*{g)}
Not valid.

\begin{prooftree}
  \assumption{\fresh{x}{X}}
  \AXC{?}
  \UIC{\fresh{y}{Y}}
  \UIC{\fresh{x}{(x \: y) \cdot Y}}
  \BIC{\fresh{x}{(X, (x \: y) \cdot Y)}}
  \UIC{\fresh{x}{+\!\!(X, (x \: y) \cdot Y)}}
  \AXC{ \vdots }
  \BIC{\eqa
    {\lam{x}{+\!\!(X,Y)}}
    {\lam{y}{+\!\!(X, (x \: y) \cdot Y)}}
  }
\end{prooftree}

Here we omit the right branch of the proof. The invalidity of the
judgement arises from the freshness requirement $\fresh{y}{Y}$.
 
\end{multicols}

\newpage
\subsection*{h)}
Valid.

\begin{prooftree}
  \assumption{\fresh{x}{X}}
  \AXC{}
  \UIC{\fresh{x}{y}}
  \UIC{\fresh{x}{\abs{y}{y}}}
  \UIC{\fresh{x}{\lam{y}{y}}}
  \BIC{\fresh{x}{(X,\abs{y}{y})}}
  \UIC{\fresh{x}{app(X,\abs{y}{y})}}
  \assumption{\fresh{x}{X}}
  \assumption{\fresh{y}{X}}
  \BIC{\eqa{X}{(x \: y) \cdot X}}
  \AXC{}
  \UIC{\eqa{y}{y}}
  \UIC{\eqa{\abs{y}{y}}{\abs{y}{y}}}
  \UIC{\eqa{\lam{y}{y}}{\lam{y}{y}}}
  \BIC{\eqa{(X,\lam{y}{y})}{((x \: y) \cdot X, \lam{y}{y})}}
  \UIC{\eqa{app(X,\lam{y}{y})}{app ((x \: y) \cdot X, \lam{y}{y})}}
  \RightLabel{$\equiv$}
  \UIC{\eqa{app(X,\lam{y}{y})}{(x \: y) \cdot app (X, \lam{y}{y})}}
  \BIC{\eqa
    {\abs{x}{app(X, \lam{y}{y})}}
    {\abs{y}{app(X, \lam{y}{y})}}
  }
  \UIC{\eqa
    {\lam{x}{app(X, \lam{y}{y})}}
    {\lam{y}{app(X, \lam{y}{y})}}
  }
\end{prooftree}


Again we used an $\equiv$ step for clarity.  $(x \: y) \cdot app (X,
\lam{y}{y})}$ is NOT a nominal term, the expression is
  syntactically equal to $app ((x \: y) \cdot X, \lam{y}{y})}$.


\newpage

\section*{Exercise 2}

This is a nontrivial proof of temination since the (\algRel) relation
is not defined inductively. A common tool for proving the termination
of programs (or, of course rewriting systems) is to find a strictly
decreasing termination function that maps the state of the program
into some \emph{well-founded} set. For this exercise we use multiset
orderings. Actually I did not know about multiset orderings before,
and I only had done simple proofs of non inductive termination.
I read \cite{DBLP:conf/icalp/DershowitzM79} to solve this exercise.

The idea is that given a base order ($\prec$) over a set $A$, we
define an order $\prec_{\mathscr{P}}$ over finite multisets of $A$
where $A \prec_{\mathscr{P}} B$ if and only if we can get $A$ from $B$
substituting zero or more elements, each by a finite set of smaller
elements (according to $\prec$).  It is well known and easy to prove
(details in \cite{DBLP:conf/icalp/DershowitzM79}) that
($\prec_{\mathscr{P}}$) is a well-founded order whenever ($\prec$) is
a well-founded order.

For the (\algRel) relation we have a total of 12 rules (six rewriting
freshness relations, as many rewriting $\alpha$-equivalence
predicates). Let us name them as: $\#_1, \#_2, \#_3, \#_4,
\#_5, \#_6,\alpha_1, \alpha_2, \alpha_3, \alpha_4, \alpha_5,
\alpha_6$. When a rule is of the form $Pr \Rightarrow Pr'$ we refer
$Pr$ and $Pr'$ as LHS and RHS, respectively.

If we define some notion of size ($sz$) for $\fresh{a}{t}$ and
$\eqa{t}{u}$ and then:

$$size(Pr) = \biguplus_{p \in Pr} sz(p) $$

\noindent
It is clear that for $\#1$, $\#5$, and $\alpha_1$, $size$ is
decreasing from LHS to RHS since $size (\LHS) \subset size(\RHS)$, and
if $A \subseteq B$ then $A \prec_{\mathscr{P}} B$ in the multiset
ordering.

For most of the other rules $size$ is also decreasing if $sz$ is
monotone with respect to the structure of the terms used in the
predicates. But note that ($\#_6$) requires $sz(\fresh{a}{\pi \cdot
  X}) > sz(\fresh{a}{id \cdot X})$ (when $\pi \neq id$) and in the
other hand ($\alpha_5$) introduces a permutation, generating possibly
many suspensions, which means that it is not clear that
$sz(\eqa{\abs{b}{l}}{\abs{a}{s}}) > sz(\eqa{(a \: b) \cdot l}{s})$ if
we take a naive approach. Also, rules ($\alpha_5$) and ($\alpha_6$)
suggest that it is a good idea to consider equivalence conditions
``bigger'' than freshness conditions.

Taking all this into account, the following $sz$ function is
proposed:

\begin{align*}
  sz(\fresh{a}{t}) &= (1, \doubl{t})\\
  sz(\eqa{t}{u}) &= (2, \simpl{\eqa{t}{u}})
\end{align*}

where:

\begin{align*}
  \doubl{a}      &= 1\\
  \doubl{\abs{a}{t}} &= 1 + \doubl{t}\\
  \doubl{f t} &= 1 + \doubl{t}\\
  \doubl{(t_1, \cdots t_n)} &= \sum_{i=1}^{n} \doubl{t_i} + 1\\
  \doubl{id \cdot X} &= 1\\
  \doubl{\pi \cdot X} &= 2 \: (\text{whenever} \: \pi \neq id) 
\end{align*}

and:

\begin{align*}
  \simpl{\eqa{t}{u}} &= \simpl{t} + \simpl{u}\\
  \simpl{a}      &= 1\\
  \simpl{\abs{a}{t}} &= 1 + \simpl{t}\\
  \simpl{f t} &= 1 + \simpl{t}\\
  \simpl{(t_1, \cdots t_n)} &= \sum_{i=1}^{n} \simpl{t_i} + 1\\
  \simpl{\pi \cdot X} &= 1
\end{align*}

With this definition, note that the codomain of $sz$ is $(\{1,2\},
\mathbb{N})$, and it is well founded considering the lexicodraphic
order.

Now, we prove that $size$ is decreasing, justifying in each case what
happens from $zise(\LHS)$ to $size(\RHS)$

In ($\#_1$, $\#_5$, $\alpha_1$), as pointed before, we erase one
element. In ($\#_2$, $\#_4$) we erase a term of size $(1,n)$ and put a
term of size $(1,n-1)$, that is, smaller. The case ($\#_6$) is
similar, works thanks to our definition. Analogously for ($\alpha_3$,
$\alpha_4$) we erase an element of size $(2,n)$ to put something of
size $(2,n-2)$.  In ($\#_3$) we erase something of size $(1,n)$ to put
a finite set of elements of the form $(1,k)$ with $k<n$. Again we get
a smaller multiset. For ($\alpha_2$) we can reason the same way.  In
($\alpha_5$) we erase a term of size $(2,k)$ and put two terms of
smaller size: $(2,k-2)$ and $(1,l)$. Note that this is true since
permiutations does not change size of terms since $\simpl{\pi \cdot X}
= 1$ independendly of $\pi$.  Finally in ($\alpha_6$) we erase an
element of size $(2,n)$ to put a finite set of elements of size
$(1,\_)$, getting a smaller multiset.


Finally we conclude. The function proposed is strictly decreasing at
each step. It is defined over a well founded set, wich proves
termination.


\newpage
\section*{Exercise 3}

\begin{multicols}{2}
\subsection*{a)}

\begin{prooftree}
  \AXC{$(\lam{x}{s(X)}) \: Y$}\RightLabel{$\beta$}
  \UIC{$s(x) [x \mapsto y] $}\RightLabel{$\sigma_{s}$}
  \UIC{$s ( x [ x \mapsto y])$}\RightLabel{$\sigma_{x}$}
  \UIC{$s(Y)$}
\end{prooftree}

\subsection*{b)}
\begin{prooftree}
  \AXC{$\fresh{y}{Y}$}
  \AXC{$(\lam{x}{\lam{y}{x}}) Y$}\RightLabel{$\beta$}
  \UIC{$(\lam{y}{x}) [x \mapsto Y]$}
  \RightLabel{$\sigma_\lambda$}
  \BIC{$\lam{y} ({x} [x \mapsto Y])$}
  \RightLabel{$\sigma_x$}
  \UIC{$\lam{y} Y$}
\end{prooftree}

\subsection*{c)}
\begin{prooftree}
  \AXC{\fresh{y}{X}}
  \AXC{$(\lam{x}{X})Y$}\RightLabel{$\beta$}
  \UIC{$X [y \mapsto Y]$}\RightLabel{$\sigma_\epsilon$}
  \BIC{$X$}
\end{prooftree}


\subsection*{d)}
\begin{prooftree}
  \AXC{$\fresh{y}{Y}$}
  \AXC{$\fresh{y}{Y}$}
  \AXC{$((\lam{x}{\lam{y}{x}}) \: Y ) \: Y$}\RightLabel{$\beta$}
  \UIC{$((\lam{y}{x}) [x \mapsto Y]) \: Y$}\RightLabel{$\sigma_\lambda$}
  \BIC{$(\lam{y}{(x[x \mapsto Y]})) \: Y$}\RightLabel{$\sigma_x$}
  \UIC{$(\lam{y}{Y})\: Y$}\RightLabel{$\beta$}
  \UIC{$ Y [y \mapsto Y]$}\RightLabel{$\sigma_\epsilon$}
  \BIC{$Y$}
\end{prooftree}

\end{multicols}


\newpage
\section*{Exercise 4}

To prove that a judgement is valid we must derive a quasi-typing. Then
we check that the linearity (compatibility) property holds.

\newcommand\tj[5]{#1 \Vvdash_{#2} #3 \vdash #4 \: : \: #5 }
\newcommand\tjc[5]{#1 \Vdash_{#2} #3 \vdash #4 \: : \: #5 }

\subsection*{a)}

\newcommand\gam{\{ a : \alpha, X : \beta \}}

\begin{prooftree}
  \AXC{$\gam_a \equiv \alpha $}\RightLabel{atm}
  \UIC{$\tj{\gam}{ \emptyset }{\emptyset}{a}{\alpha}$}
  \AXC{$\gam_X \equiv X $}\RightLabel{var}
  \UIC{$\tj{\gam}{ \emptyset }{\emptyset}{X}{\beta}$}
  \RightLabel{tpl}
  \BIC{$\tj{\gam}{ \emptyset }{\emptyset}
    {(a,X)}{\alpha \times \beta}$}
\end{prooftree}

\noindent
Compatibility  property  holds  since  there is  only  one  essential
enviroment in discussion (one occurence of $X$).
Then
$\tjc{\gam}{ \emptyset }{\emptyset}
    {(a,X)}{\alpha \times \beta}$

\subsection*{b)}

\begin{prooftree}
  \AXC{}\RightLabel{atm}
  \UIC{$\tj{a : \alpha }{\emptyset}{\emptyset}{a}{\alpha}$}\RightLabel{abs}
  \UIC{$\tj{\emptyset}{\emptyset}{\emptyset}{[a]a}{[\alpha]\alpha}$}
\end{prooftree}

\noindent
Here, and from now on we use a more lightweight notation, avoiding the
brackets and the inmediate check on (atm) rules. No variables ever
occur, meaning that the compatibility property holds trivially.

\noindent
We conclude $\tjc{\emptyset}{\emptyset}{\emptyset}{[a]a}{[\alpha]\alpha}$.

\subsection*{c)}
\begin{prooftree}
  \AXC{}\RightLabel{atm}
  \UIC{$\tj{a : \alpha}{\emptyset}{\emptyset}{a}{\alpha}$}
  \RightLabel{abs}
  \UIC{$\tj{a : \beta}{\emptyset}{\emptyset}{[a]a}{[\alpha]\alpha}$}
\end{prooftree}

\noindent
Here note that $\{ a : \beta \} \bowtie \{ a : \alpha \} = \{ a :
\alpha \}$.  Again the compatibility property holds trivially.

\subsection*{d)}

\newcommand\es{\emptyset}
\newcommand\perm[2]{(#1) \cdot #2}

\begin{prooftree}
  \AXC{}\RightLabel{atm} \UIC{$\tj{a : \tau_1, b : \tau_2, X :
      \tau}{\emptyset}{\emptyset}{(a \: b) \cdot X}{\tau} $}
\end{prooftree}

\noindent
Again the compatibility property is trivial since the variable $X$
occurs only once.  This is (counterintuitively, at least) true even
when atoms $a$ and $b$ have different types.


\subsection*{e)}
\label{sec:e}

\begin{prooftree}
  \AXC{}\RightLabel{atm}
  \UIC{$\tj{\Gamma_1 \equiv a : \tau_1, b : \tau_1, X : \tau}
    {\es}{\es}{\perm{a \: b}{X}}{\tau}$}
  \AXC{}\RightLabel{atm}
  \UIC{$\tj{\Gamma_2 \equiv a : \tau_1, b : \tau_1, X : \tau}
    {\es}{\es}{id \cdot X}{\tau}$}\RightLabel{tpl}
  \BIC{$\tj{a : \tau_1, b : \tau_1, X : \tau}
    {\es}{\es}{(\perm{a \: b}{X}, id \cdot X)}{\tau \times \tau}$}
\end{prooftree}

\noindent
The compatibility property holds if $\Gamma_1^{(a \: b)^{-1}} \equiv
\Gamma_2$ (recall that $\Sigma \equiv \es$, so we ignore it) which
holds since $a$ and $b$ have both type $\tau_1$.  Then $\tjc{a :
  \tau_1, b : \tau_1, X : \tau} {\es}{\es}{(\perm{a \: b}{X}, id \cdot
  X)}{\tau \times \tau}$.  Note that if the case were similar to the
previous one, where $a$ and $b$ have different types then the
essential environments would be different, meaning that the
quasi-typing is not enough to show the validity of the judgement.


\subsection*{f)}

In this case, let us develop a piece of the quasi-typing derivation, the
branch marked with ``?'' is not yet closed:

\begin{prooftree}
  \AXC{?}
  \UIC{$\tj{X :
      \tau}{\es}{\fresh{a}{X}}{[a] id \cdot X}{\tau}$}
  \AXC{}
  \RightLabel{var}
  \UIC{$\tj{X :
      \tau}{\es}{\fresh{a}{X}}{id \cdot X}{\tau}$}
  \RightLabel{tpl}
  \BIC{$\tj{X :
      \tau}{\es}{\fresh{a}{X}}{(\abs{a}{id \cdot X},id \cdot X)}
    {(\tau\times\tau)}$}
\end{prooftree}

We are stuck here, since the type of $[a] id \cdot X$ must be an
abstraction to follow up the proof development.  If the quasi-typing
is not derivable then the judgement cannot be valid. As it is
presented in the slides (``show that each of these typing judgements
is valid'') it is assumed that all judgements are valid.  In
\cite{Fairweather:2018:TNR:3176362.3161558} this is also an example
of a valid judgement. Perhaps there is a typo?

Assuming that the type of $([a]id \cdot X, id \cdot X)$ is something
like $([\gamma]\tau\times\tau)$ then we have two (var) rules proving
$id \cdot X : \tau$ with different typing contexts (one with
information about $a$, and one without). Since $\fresh{a}{X}$ is an
assumption in the freshness context we do not care about $a$ and both
occurrences of $X$ are typed in the same essential environment.


\subsection*{g)}


\begin{prooftree}
  \AXC{}
  \RightLabel{var}
  \UIC{$\tj{a : \beta, b : \beta, X : \tau}{\es}{\es}{(a \:
      b) \cdot X}{\tau}$}
  \AXC{}
  \RightLabel{var}
  \UIC{$\tj{a : \beta, b : \beta, X :
      \tau}{\es}{\es}{id \cdot X}{\tau}$}
  \RightLabel{tpl}
  \BIC{$\tj{a : \beta, b :
      \beta, X : \tau}{\es}{\es}{((a \: b) \cdot X, id \cdot
      X)}{\tau\times\tau}$}\RightLabel{abs} \UIC{$\tj{a : \alpha, b :
      \beta, X : \tau}{\es}{\es}{[a]((a \: b) \cdot X, id \cdot
      X)}{[\beta]\tau\times\tau}$}
\end{prooftree}

\noindent
Like in (e) both occurrences of $X$ are typed in the same essential
environment because $a$ and $b$ have the same type. This is true
because after applying the (abs) rule (reading bottom-up) the typing
context is updated.


\newpage
\section*{Exercise 5}
They are not all typeable closed. Let us look at the fourth and last
rule.  If $sub([a](lam[b]X),Z) : \alpha' \Rightarrow\alpha$ then since
$sub : \forall(\alpha,\beta)\langle([\alpha]\beta\times\alpha)
\hookrightarrow \beta\rangle$ to get the right type it must be
instanced with $\beta := \alpha' \Rightarrow\alpha$. To see it
clearer:

{\small
\begin{prooftree}
  \AXC{$\forall(\alpha,\beta)\langle([\alpha]\beta\times\alpha)
    \hookrightarrow \beta \rangle \succcurlyeq \langle
    ([\gamma](\alpha' \Rightarrow \alpha)\times \gamma)\hookrightarrow
    \alpha' \Rightarrow \alpha \rangle $} \AXC{$\vdots$}
  \UIC{$\tj{X:\alpha, Z:\gamma}{\Sigma}{\fresh{b}{Z}}{([a](lam[b]X),Z)
    }{ ([\gamma](\alpha' \Rightarrow \alpha)\times \gamma)}$} \BIC{$\tj{X:\alpha,
      Z:\gamma}{\Sigma}{\fresh{b}{Z}}{sub([a](lam[b]X),Z) }{ \alpha'
      \Rightarrow\alpha}$}
\end{prooftree}
}

\noindent
This violates the condition (c).

\newpage
\section*{Exercise 6}

\newcommand\sol[2]{\langle #1, #2 \rangle}
\newcommand\eqq{\overset{\curlywedge}{\approx}_\alpha}

Recall that a solution for $Pr$ is a pair $\sol{\Phi}{\sigma}$ such that
\begin{enumerate}
  \item $\Phi \vdash \pi \curlywedge t \sigma$ for all
    $\pi\curlywedge^{?} t \in Pr$
  \item
    $\Phi \vdash s\sigma \eqq t \sigma$ for all
    $s \eqq^{?} t \in Pr$
  \item
    $X \sigma = X \sigma\sigma$ for $X \in \textt{\mathtt{Var}(Pr)}$
\end{enumerate}

This was taken from \cite{DBLP:conf/rta/Ayala-RinconFN18}. (I think
that this information was not part of the slides)

Now, to prove that simplification rules preserves results we analize
each case. Most of them are similar, a couple of cases are proved in
\cite{DBLP:conf/rta/Ayala-RinconFN18} with no equational theories. For
this case they are similar. The new cases are the ones involving $+$,
and can be studied the same way (for one inclusion applying a rule,
for the other an inversion lemma).

For instance, for the case $Pr \uplus \{ \pi \curlywedge^{?} +(s,t) \}
\Rightarrow Pr \uplus \{ \pi \cdot s \eqq s, \pi \cdot t \eqq t\}$ if
$\sol{\Phi}{\sigma}$ is a solution of $Pr \uplus \{ \pi
\curlywedge^{?} +(s,t) \}$ in particular conditions 1,2 are trivially
satisfied for all constraints in $Pr$ (and 3, of course). We also know
that $\Phi \vdash \pi \curlywedge +(s,t)\sigma$ meaning that
$\Phi \vdash \pi \curlywedge +(s\sigma,t\sigma)$. From the rule:


\begin{prooftree}
  \AXC{$\Phi \vdash \pi \cdot t_0 \approx t_0$}
  \AXC{$\Phi \vdash \pi \cdot t_1 \approx t_1$}
  \BIC{$\Phi \vdash \pi \curlywedge +(t_0, t_1)$}
\end{prooftree}

applied with $t_0 = s$, $t_1 = t$ and inversion we get exactly that
$\Phi \vdash \pi \cdot s\sigma \approx s\sigma$ and $\Phi \vdash \pi
\cdot t\sigma \approx t\sigma$, then $\sol{\Phi}{\sigma}$ is solution of
$Pr \uplus \{ \pi \cdot s \eqq s, \pi \cdot t \eqq t\}$.
The other inclusion is similar applying the rule forwards.

For $Pr \uplus \{ \pi \curlywedge^{?} +(s,t) \}
\Rightarrow Pr \uplus \{ \pi \cdot s \eqq t, \pi \cdot t \eqq s\}$
we reason the same way using the rule:

\begin{prooftree}
  \AXC{$\Phi \vdash \pi \cdot t_0 \approx t_1$}
  \AXC{$\Phi \vdash \pi \cdot t_1 \approx t_0$}
  \BIC{$\Phi \vdash \pi \curlywedge +(t_0, t_1)$}
\end{prooftree}

For the two cases with $\alpha$-equalities we also reason the same way
using the rules for $\alpha$-equality.

\newpage
\bibliographystyle{alpha} \bibliography{bib}

\end{document}
